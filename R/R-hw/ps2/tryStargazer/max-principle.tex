

\documentclass[11pt]{article}

\usepackage{geometry,amsmath,amssymb,xfrac,amsthm} 
%\usepackage [usenames,dvipsnames]{color}
\def\ds{\displaystyle}
\def\ul{\underline}
\def\mb{\mathbf}
\usepackage{graphicx}
\usepackage{tikz}
\oddsidemargin=-0.0cm
\evensidemargin=-0.0cm
\topmargin=-0.0cm
\textwidth=7in
\textheight=9in
\newtheorem{theorem}{Theorem}
\newtheorem{corollary}{Corollary}
\newtheorem{lemma}{Lemma}
\newtheorem{remark}{Remark}
\def\ds{\displaystyle}
\def\ul{\underline}
\def\mb{\mathbf}
\usepackage{framed,color}
\definecolor{shadecolor}{rgb}{1,0.8,0.3}
\usepackage{fancybox}
\usepackage{mdframed}

%%%%%%%%%%%%%%%%%%%%%%%%%%%%%%%%%%%%%%%%%%%%%%%%%%%%%%%%%%%%%%%%%%

\title{Easy Maximum Principle}
\pagestyle{myheadings}
\author{}
\date{}
\def\ds{\displaystyle}
\def\ul{\underline}

\begin{document}
\maketitle
%\markboth{final}{final}
%{\color{red} stuff}

\begin{theorem}
Suppose $u$ is bounded and harmonic in a bounded open set $\Omega$.
\newline  
Suppose $\ds \limsup_{z\to p \in \partial \Omega}u(z) \le M$ for all $p\in \partial \Omega$.  Then $u(z) \le M$.
\end{theorem}

\begin{proof}

Let $\delta >0$. Let $W=\{z: z\in \Omega, u(z)>M+\delta\} \subset \Omega$. Then $\overline{W} \subset \overline{\Omega}$, so $\partial W \subset \Omega \cup \partial \Omega$. First we prove that $\partial W \cap \partial \Omega = \emptyset$. For if $p\in \partial \Omega$ then $\limsup_{z\to p \in \partial \Omega}u(z) \le M$.  But if $p\in \partial W$ then 
$\limsup_{z\to p \in \partial W}u(z) \ge M+\delta$. So $\overline{W} \subset \Omega$ and $u$ is continuous on $\overline{W}$. Now a point $p$ in $\partial W$ must contain points in $W$ and points not in $W$. Since $u$ is continuous, if $u(p)>M+\delta$, so are nearby points ($W$ is open). So it must be the case that $u(p) \le M+\delta$.  In other words $u(p) \le M+\delta$ for points $p\in \partial W$. By the maximum principle $u(p) \le M+\delta$ for all $p\in W$.  This contradicts the definition of $W$.  So $W=\emptyset$. In other words $u(p)\le M+\delta$ for all $p\in \Omega$.  Since $\delta >0$ is arbitrary, $u(p) \le M$ for all $p \in \Omega$.
\end{proof}

\begin{equation}
\Upsilon_{\max_{k \in lp(i)}(L_k)} = 1
\end{equation}

Inline math, i.e. math within text: $\min_{n}\Arrowvert x \Arrowvert_{1}$
A displayed formula:
\[\min_{n}\Arrowvert x \Arrowvert_{1}\]
 
Inline math, but in a displayed style:
$\displaystyle\min_{n}\Arrowvert x \Arrowvert_{1}$
 
Inline math, only \verb|\min| in a displayed style, i.e. limits below:
$\min\limits_{n}\Arrowvert x \Arrowvert_{1}$

%\bibliographystyle{amsplain}
%\bibliography{header}

\end{document}
